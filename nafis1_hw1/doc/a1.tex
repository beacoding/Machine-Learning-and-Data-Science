\documentclass{article}

\usepackage{fullpage}
\usepackage{color}
\usepackage{amsmath}
\usepackage{url}
\usepackage{verbatim}
\usepackage{graphicx}
\usepackage{parskip}
\usepackage{amssymb}
\usepackage{listings} % For displaying code

% Colors
\definecolor{blu}{rgb}{0,0,1}
\def\blu#1{{\color{blu}#1}}
\definecolor{gre}{rgb}{0,.5,0}
\def\gre#1{{\color{gre}#1}}
\definecolor{red}{rgb}{1,0,0}
\def\red#1{{\color{red}#1}}
\def\norm#1{\|#1\|}
\newcommand{\argmin}[1]{\mathop{\hbox{argmin}}_{#1}}
\newcommand{\argmax}[1]{\mathop{\hbox{argmax}}_{#1}}
\def\R{\mathbb{R}}
\newcommand{\fig}[2]{\includegraphics[width=#1\textwidth]{#2}}
\newcommand{\centerfig}[2]{\begin{center}\includegraphics[width=#1\textwidth]{#2}\end{center}}
\def\items#1{\begin{itemize}#1\end{itemize}}
\def\enum#1{\begin{enumerate}#1\end{enumerate}}
\def\argmax{\mathop{\rm arg\,max}}
\def\argmin{\mathop{\rm arg\,min}}


\begin{document}

\title{CPSC 340 Assignment 1 (due 2017-01-22 at 11:59pm)}
\author{Data Exploration, Decision Trees, Training and Testing, Naive Bayes}
\date{}
\maketitle

\textbf{IMPORTANT!!!!! Before proceeding, please carefully read the general homework instructions at} \url{https://github.ubc.ca/cpsc340/home/blob/master/homework_instructions.md}.

\textbf{A note on the provided code:} in the \emph{code} directory we provide you with a file called
\emph{main.py}. This file, when run with different arguments, runs the code for different
parts of the homework assignment. For example,
\begin{verbatim}
python main.py -q 1.1
\end{verbatim}
runs the code for Question 1.1. At present, this should do nothing, because the code
for Question 1.1 still needs to be written (by you). But we do provide some of the bits
and pieces to save you time, so that you can focus on the machine learning aspects. For example, you'll see that the provided code already loads the datasets for you. The file \emph{utils.py} contains some helper functions. You don't need to understand or modify the code in there. To complete your assignment, you will need to modify \emph{main.py}, \emph{decision\_stump.py}, and \emph{naive\_bayes.py}.

We have tried to provide high-qualty code. However, if we come across any bugs in the provided code,
we will send you a pull request to fix it. You will then receive a notification from GitHub.

\vspace{1em}
We use \blu{blue} to highlight the deliverables that you must answer/do/submit with the assignment.

\section{Data Exploration}

Your repository contains the file \emph{fluTrends.pkl}, which contains estimates of the influenza-like illness percentage over 52 weeks on 2005-06 by Google Flu Trends.
Your \emph{main.py} loads this data for you and stores it in the 2-D numpy array \texttt{X}, where each row corresponds to a week and each column corresponds to a different region. The names of the columns are loaded into the variable \texttt{names}.\footnote{If you are already a Python user, you may be wondering why we aren't using Pandas. This is certainly a great library and you are welcome to convert the data to a Pandas DataFrame if you wish. But my current plan is to avoid Pandas in this course, in order to reduce the number of libraries that you are required to learn.}

\subsection{Summary Statistics}

\blu{Report the following statistics}:
\enum{
\item The minimum, maximum, mean, median, and mode of all values across the dataset.
\item The $10\%$, $25\%$, $50\%$, $75\%$, and $90\%$ quantiles across the dataset.
\item The regions with the highest and lowest means, and the highest and lowest variances.
\item The pairs of regions with the highest and lowest correlations.
}
In light of parts 1 and 2, \blu{is the mode a reliable estimate of the most ``common" value? Describe another way we could give a meaningful ``mode" measurement for this (continuous) data.}


\subsection{Data Visualization}

\blu{Show the following figures}:
\enum{
\item A plot containing the weeks on the $x$-axis and the percentages for each region on the $y$-axis.
\item A boxplot grouping data by weeks, showing the distribution across regions for each week.
\item A histogram showing the distribution of each the values in the matrix $X$.
\item A single histogram showing the distribution of \emph{each} column in $X$.
\item A scatterplot between the two regions with lowest correlation.
\item A scatterplot between the two regions with highest correlation.
}

Note: this is the first time we are asking you to include a plot with your submission. Please refer to the ``Figures'' section of the homework instructions document.

I have embedded the CPSC 340 logo below, so you have an example of how to embed a figure in LaTeX.

This figure:
\begin{center}
\end{center}
was embedded with this code:
\begin{verbatim}
\begin{center}
\fig{0.1}{../figs/cpsc340-logo}
\end{center}
\end{verbatim}

The 0.1 represents the desired width of the image. The centering commands are for centering the image in the page; you can try removing them to see what happens. The \texttt{fig} command is a custom command defined at the top of this file, which calls the \texttt{includegraphics} command. This comes from the \texttt{graphicx} package which is loaded at the top of this file with the \texttt{usepackage} command. Finally, as far as I can tell, including the file extension is optional when including an image in LaTeX. In the above, if you replace \texttt{cpsc340-logo} with \texttt{cpsc340-logo.png} it should work exactly the same way.

\section{Decision Trees}

If you run \texttt{python main.py -q 2.1}, it will load a dataset containing longtitude and latitude data for 400 cities in the US, along with a class label indicating whether they were a ``red" state or a ``blue" state in the 2012 election.\footnote{The cities data was sampled from \url{http://simplemaps.com/static/demos/resources/us-cities/cities.csv}. The election information was collected from Wikipedia.}
Specifically, the first column of the variable $X$ contains the longitude and the second variable contains the latitude,
while the variable $y$ is set to $1$ for blue states and $2$ for red states.
After it loads the data, it plots the data and then fits two simple classifiers: a classifier that always predicts the
most common label ($1$ in this case) and a decision stump that discretizes the features (by rounding to the nearest integer)
and then finds the best equality-based rule (i.e., check if a feature is equal to some value).
It reports the training error with these two classifiers, then plots the decision areas made by the decision stump.

\subsection{Decision Stump Implementation}

The file \emph{decision\_stump.py} contains the functions \texttt{fit\string_equality} and \texttt{predict\string_equality} which finds the best decision stump using the equality rule and then makes predictions using that rule. Instead of discretizing the data and using a rule based on testing an equality for a single feature, we want to check whether a feature is above or below a threshold and split the data accordingly (as discussed in class). \blu{Fill in the \texttt{fit} and \texttt{predict} functions to do this, and report the updated error you obtain by using inequalities instead of discretizing and testing equality.}

Hint: you may want to start by copy/pasting the contents \texttt{fit\string_equality} and \texttt{predict\string_equality} and making modifications from there. %You can make modifications from there. % But you are asked to keep them separate rather than directly modifying the provided code so that you can keep both versions in tact and compare them.
% Note: please keep the same variable names, as subsequent parts of this assignment rely on this!


\subsection{Constructing Decision Trees}

Once your decision stump is finished, the code in \texttt{decision\string_tree} will be able to fit a decision tree of depth 2 to the same dataset (which results in a lower training error). Look at how the decision tree is stored and how the (recursive) \emph{predict} function works. \blu{Using the same splits as the fitted depth-2 decision tree, write an alternative version of the predict function for classifying one training example as a simple program using if/else statements (as in slide 9 of the Decision Trees lecture).} Save your code in a new file called \texttt{simple\string_decision.py} (in the \emph{code} directory) and make sure you link to this file from your README.

% Hint: look at the \texttt{model} returned by \emph{fit}.
% Hint: if you want to stop the decision tree's execution at some point, you can \texttt{import pdb} (Python debugger) at the top of the file and call \texttt{pdb.set\string_trace()} to set a breakpoint somewhere in the code. From there you can look at the values of the variables.

Modify the \emph{depth} variable in this demo to fit deeper decision trees. Notice that the error stops decreasing after a certain depth.
In contrast, if you call the scikit-learn version (see code) you will notice that the error eventually decreases to 0.
\blu{Why does the training error stop decreasing when you use classification accuracy?}


\subsection{Cost of Fitting Decision Trees}

In class, we discussed how in general the decision stump minimizing the classification error can be found in $O(nd\log n)$ time. Using the greedy recursive splitting procedure, \blu{what is the total cost of fitting a decision tree of depth $m$ in terms of $n$, $d$, and $m$?}

Hint: even thought there could be $(2^m-1)$ decision stumps, keep in mind not every stump will need to go through every example. Note also that we stop growing the decision tree if a node has no examples, so we may not even need to do anything for many of the $(2^m-1)$ decision stumps.

\gre{The cost at each depth is O(dnlogn) as sorting costs O(nlogn) per feature.We look at the examples at each depth.To go through a depth of m we have total cost of O(mndlogn).}


\section{Training and Testing}
If you run \texttt{python main.py \string-q 3.1}, it will load \emph{citiesSmall.pkl}.
Note that this file contains not only training data, but also test data, \texttt{X\string_test} and \texttt{y\string_test}.
After training a depth-2 decision tree it will evaluate the performance of the classifier on the test data.\footnote{The code uses the "information gain" splitting criterion; see the Decision Trees bonus slides for more information.} With a depth-2 decision tree, the training and test error are fairly close, so the model hasn't overfit much.

\subsection{Training and Testing Error Curves}

\blu{Make a plot that contains the training error and testing error as you vary the depth from 1 through 15. How do each of these errors change with the decision tree depth?}

\gre{Test error in green and training error in blue both goes down. The training error goes down to 0 but test error does not keep going down after depth 10.}

\subsection{Validation Set}

Suppose that we didn't have an explicit test set available. In this case, we might instead use a \emph{validation} set. Split the training set into two equal-sized parts: use the first $n/2$ examples as a training set and the second $n/2$ examples as a validation set (we're assuming that the examples are already in a random order). \blu{What depth of decision tree would we pick to minimize the validation set error? Does the answer change if you switch the training and validation set? How could use more of our data to  estimate the depth more reliably?}

\gre{We will pick a depth of 3 if validation error is minimized.Yes the answer changes to depth 6 if we switch the training and validation set.We can do n-fold cross validation to add more reliability in our result. }




\section{Naive Bayes}

In this section we'll first review a standard way that Bayes rule is used and then explore implementing a naive Bayes classifier, a very fast classification method that is often surprisingly accurate for text data with simple representations like bag of words.


\subsection{Bayes rule for drug testing}

Suppose a drug test produces a positive result with probability $0.99$ for drug users, $P(T=1|D=1)=0.99$. It also produces a negative result with probability $0.99$ for non-drug users, $P(T=0|D=0)=0.99$. The probability that a random person uses the drug is $0.001$, so $P(D=1)=0.001$.

\blu{What is the probability that a random person who tests positive is a user, $P(D=1|T=1)$?}

\gre{$P(D=1|T=1)$ = $P(T=1|D=1)$*$P(D=1)$/$P(T=1|D=1)P(D=1) +(T=1|D=1)P(D=0)$}
\gre{$.99*.001/(.99)(.001)+(.01)(.999)$}



\subsection{Naive Bayes by Hand}

Consider the dataset below, which has $10$ training examples and $2$ features:
\[
X = \begin{bmatrix}0 & 1\\1 & 1\\ 0 & 0\\ 1 & 1\\ 1 & 1\\ 0 & 0\\  1 & 0\\  1 & 0\\  1 & 1\\  1 &0\end{bmatrix}, \quad y = \begin{bmatrix}1\\1\\1\\1\\1\\1\\0\\0\\0\\0\end{bmatrix}.
\]
Suppose you believe that a naive Bayes model would be appropriate for this dataset, and you want to classify the following test example:
\[
\hat{x} = \begin{bmatrix}1 & 1\end{bmatrix}.
\]

\blu{(a) Compute the estimates of the class prior probabilities:}
\items{
\item$ p(y = 1)$ =\gre {6/10}
\item $p(y = 0)$ \gre {1-(6/10) = 4/10}
}

\blu{(b) Compute the estimates of the 4 conditional probabilities required by naive Bayes for this example:}
\items{
\item $p(x_1 = 1 | y = 1)$ =\gre {3/6}
\item $p(x_2 = 1 | y = 1)$ =\gre {4/6}
\item $p(x_1 = 1 | y = 0)$ =\gre {4/4}
\item $p(x_2 = 1 | y = 0)$ =\gre {1/4}
}

\blu{(c) Under the naive Bayes model and your estimates of the above probabilities, what is the most likely label for the test example? (Show your work.)}
\gre {If $p(y = 0|x_1 = 1, x_2 = 1) > p(y = 1|x_1 = 1, x_2 = 1)$ then the likely label is 0 otherwise its 1.}

\gre{$p(y = 1|x_1 = 1, x_2 = 1)$}

\gre{$=p(x_1 = 1, x_2 = 1)|p(y=1)$}\red{We do not require the denominator as we  }

\gre{$=p(x_1=1|y=0)p(x_2 = 1|y=0)p(y=1)$}
\gre{$.5*.66*.6$}

\gre{$p(y = 0|x_1 = 1, x_2 = 1)$}

\gre{$=p(x_1 = 1, x_2 = 1)|p(y=0)$}\red{We do not require the denominator as we  }

\gre{$=p(x_1=1|y=0)p(x_2 = 1|y=0)p(y=0)$}
\gre{.25 *.4}
 
\gre{More likely label is 1 as its probability is higher than that of 0} 





\subsection{Naive Bayes Implementation}

If you run \texttt{python main.py \string-q 4.3} it will first load the following dataset:
\enum{
\item $groupnames$: The names of four newsgroups.
\item $wordlist$: A list of words that occur in posts to these newsgroups.
\item $X$: A binary matrix. Each row corresponds to a post, and each column corresponds to a word from the word list. A value of $1$ means that the word occured in the post.
\item $y$: A vector with values $1$ through $4$, with the value corresponding to the newsgroup that the post came from.
\item $Xvalidate$ and $yvalidate$: the word lists and newsgroup labels for additional newsgroup posts.
}
It will train a decision tree of depth 20 and report its validation error, followed by training a naive Bayes model.

While the \emph{predict} function of the naive Bayes classifier is already implemented, the calculation of the variable $p\_xy$ is incorrect (right now, it just sets all values to $1/2$). \blu{Modify this function so that \emph{p\_xy} correctly computes the conditional probability of these values based on the frequencies in the data set. Hand in your code and report the validation error that you obtain.}


\subsection{Runtime of Naive Bayes for Discrete Data}

Assume you have the following setup:
\items{
\item The training set has $n$ objects each with $d$ features.
\item The test set has $t$ objects with $d$ features.
\item Each feature can have up to $c$ discrete values (you can assume $c \leq n$).
\item There are $k$ class labels (you can assume $k \leq n$)
}
You can implement the training phase of a naive Bayes classifier in this setup in $O(nd)$, since you only need to do a constant amount of work for each $X(i,j)$ value. (You do not have to actually implement it in this way for the previous question, but you should think about how this could be done). \blu{What is the cost of classifying $t$ test examples with the model?}

\gre{There are 3 for loops : one for d features ,one for T objects and one for k class labels.  All of the loops are doing equal amount of work therefore O(dtk).}

% Making decisions with different costs

\end{document}
